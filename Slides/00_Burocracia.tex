\pdfoptionpdfminorversion=5
%\documentclass[spanish,professionalfonts]{beamer}
\documentclass[spanish]{beamer}
% \usepackage{lmodern}
\usepackage[utf8]{inputenc}
\usepackage[spanish,mexico]{babel}
\usepackage{amsmath}
\usepackage{amssymb}
\usepackage{color}
\usepackage{xmpmulti}
\usefonttheme{professionalfonts}
\usefonttheme{serif} % default family is serif
% \usepackage{fontspec}
% \setmainfont{Liberation Serif}

% \usefonttheme[onlymath]{serif}
\usepackage{tabularx} 
%%Regular colours! 
\definecolor{MyBlack}{RGB}{0,0,0}
\definecolor{MyRed}{RGB}{255,30,30}
\definecolor{MyBlue}{RGB}{80,100,255}
\definecolor{MyDarkBlue}{RGB}{0,0,100}
\definecolor{MyGreen}{RGB}{0,255,0}
\definecolor{MyCyan}{RGB}{0,255,255}
\definecolor{MyBrown}{RGB}{180,90,10}
\definecolor{MyPurple}{RGB}{160,20,255}
\definecolor{MyDarkPurple}{RGB}{80,0,120}
\definecolor{MyLightPurple}{RGB}{180,70,255}
\definecolor{MyYellow}{RGB}{255,255,0}
\definecolor{MyOrange}{RGB}{255,160,10}
\definecolor{MyGray}{RGB}{170,170,170}
\definecolor{MyDarkGray}{RGB}{15,25,30}
\definecolor{MyWhite}{RGB}{255,255,255}
\definecolor{MyInvisible}{RGB}{0,0,0}
\definecolor{MyPink}{RGB}{255,0,120}

\usetheme{Boadilla}
\usecolortheme{albatross}


% \setbeamercovered{dynamic}
% \useoutertheme{shadow}
\useinnertheme{rectangles}
\usecolortheme[MyPurple]{structure}
\setbeamercolor{structure}{bg=MyDarkGray, fg=MyLightPurple}
\setbeamercolor{frametitle}{bg=MyDarkGray, fg=MyBlue}
\setbeamercolor{title}{bg=MyDarkPurple, fg=MyWhite}
\setbeamercolor{normal text}{bg=MyBlack,fg=MyWhite}
% \setbeamercolor{alerted text}{fg=orange}
\setbeamercolor{block}{bg=MyPink,fg=MyOrange}
\setbeamercolor{background canvas}{bg=MyBlack,fg=MyDarkGray}
% \setbeamercolor{titlelike}{fg=MyBrown}
% \setbeamercolor{section in sidebar}{fg=MyRed}
% \setbeamercolor{section in sidebar shaded}{fg= grey}
% \setbeamercolor{subsection in sidebar}{fg=blue}
% \setbeamercolor{subsection in sidebar shaded}{fg= grey}
% \setbeamercolor{sidebar}{bg=red}


\setbeamercolor{block body}{bg=MyDarkGray}
\setbeamercolor{block title}{bg=MyDarkBlue, fg=MyGreen}

\setbeamertemplate{navigation symbols}{}%remove navigation symbols

\def\tcr#1{\textcolor{MyRed}{#1}}
\def\tcb#1{\textcolor{MyBlue}{#1}}
\def\tcbck#1{\textcolor{MyBlack}{#1}}
\def\tcg#1{\textcolor{MyGreen}{#1}}
\def\tcgr#1{\textcolor{MyGray}{#1}}
\def\tcbr#1{\textcolor{MyBrown}{#1}}
\def\tcp#1{\textcolor{MyPurple}{#1}}
\def\tcy#1{\textcolor{MyYellow}{#1}}
\def\tco#1{\textcolor{MyOrange}{#1}}
\def\tcw#1{\textcolor{MyWhite}{#1}}
\def\tci#1{\textcolor{MyInvisible}{#1}}
\def\tcpk#1{\textcolor{MyPink}{#1}}
\def\tcc#1{\textcolor{MyCyan}{#1}}

\def\NS{{\mathrm{NoSplit}}}
\def\ff{{\mathrm{MaxChoiceCols}}}
\def\Av{{\mathrm{Avoid}}}
\def\I{{\mathcal{I}}}
\newcommand{\ncols}[1]{\| #1 \|}
\def\qed{ \hskip 20pt{$\blacksquare$}\hfil}
\def\forb{{\mathrm{forb}}}
\def\cA{{\mathcal{A}}}
\def\cB{{\mathcal{B}}}
\def\cC{{\mathcal{C}}}
\def\cD{{\mathcal{D}}}
\def\cE{{\mathcal{E}}}
\def\cF{{\mathcal{F}}}
\def\cG{{\mathcal{G}}}
\def\cH{{\mathcal{H}}}
\def\cI{{\mathcal{I}}}
\def\cJ{{\mathcal{J}}}
\def\cK{{\mathcal{K}}}
\def\cL{{\mathcal{L}}}
\def\cM{{\mathcal{M}}}
\def\cN{{\mathcal{N}}}
\def\cO{{\mathcal{O}}}
\def\cP{{\mathcal{P}}}
\def\cQ{{\mathcal{Q}}}
\def\cR{{\mathcal{R}}}
\def\cS{{\mathcal{S}}}
\def\cT{{\mathcal{T}}}
\def\cU{{\mathcal{U}}}
\def\cV{{\mathcal{V}}}
\def\cW{{\mathcal{W}}}
\def\cX{{\mathcal{X}}}
\def\cY{{\mathcal{Y}}}
\def\cZ{{\mathcal{Z}}}

\def\I{{\mathcal{I}}}

\def\CC{{\mathbb{C}}}
\def\HH{{\mathbb{H}}}
\def\RR{{\mathbb{R}}}
\def\NN{{\mathbb{N}}}
\def\ZZ{{\mathbb{Z}}}
\def\QQ{{\mathbb{Q}}}

\newtheorem{conjecture}{Conjetura}
\newtheorem{problema}{Problema}
\newtheorem{teorema}{Teorema}
\newtheorem{proposicion}{Proposición}
\newtheorem{ejercicio}{Ejercicio}
\newtheorem{definicion}{Definición}
\newtheorem{observacion}{Observación}
\newtheorem{lema}{Lema}
\newtheorem{ejemplo}{Ejemplo}

\title{Burocracia}
\author[M. Raggi (\texttt{mraggi@gmail.com})]{Miguel Raggi \\ \texttt{\url{mraggi@gmail.com}}}
\institute[]{\tcr{Re}\tco{de}\tcy{s N}\tcg{eu}\tcc{ron}\tcb{al}\tcp{es} \\ Escuela Nacional de Estudios Superiores \\ UNAM}

\begin{document}

% \AtBeginSection[]
% {
% \begin{frame}
% \frametitle{Índice:}
% \tableofcontents[currentsection]
% \end{frame} 
% }

% \begin{frame}\frametitle{Color Test}
% \tcw{blanco}, \tcb{azul}, \tcr{rojo}, \tcg{verde}, \tcp{morado}, \tcy{amarillo}, \tcbr{café}, \tcbck{negro}, \tcgr{gris}, \tco{anaranjado}, \tci{Invisible}
% \end{frame}

\begin{frame}
\titlepage
\end{frame}



% \begin{frame}
% \frametitle{Índice:}
% \tableofcontents
% \end{frame} 



\section{Funcionamiento del curso}
\begin{frame}\frametitle{2020, el mejor año sin duda}
\begin{center}
	\huge{¡Bienvenidos a \tcr{Re}\tco{de}\tcy{s N}\tcg{eu}\tcc{ron}\tcb{al}\tcp{es}!}
	\pause
	
	...\textit{en linea}!
\end{center}
\end{frame}

\begin{frame}\frametitle{\tcr{Re}\tco{de}\tcy{s N}\tcg{eu}\tcc{ron}\tcb{al}\tcp{es}}
\begin{itemize}
 \item ¡Este es el curso \tcg{más importante} de la carrera!\pause
 \item Si me conocen, saben que en general mis cursos no están diseñados para que todos saquen 10. \pause
 \item Este es diferente: \tcg{todos deben sacar 10}.\pause
 \item Eso no significa para nada que será fácil. Probablemente será el curso que requiera más trabajo de la carrera, pero no será trabajo intelectualmente difícil.
\end{itemize}
\end{frame}


\begin{frame}\frametitle{¿Cómo va a funcionar esto?}
\begin{enumerate}
 \item 100\% en linea. \pause
 \item Principalmente será \tcg{asíncrono} con actividades como:
 \begin{itemize}
    \item Videos cortos ($\sim$5-10 mins)
    \item Quizzes sencillos de opción múltiple sobre los videos
    \item Tareas
 \end{itemize}\pause
 \item Al terminar cada tema tendremos una sesión \tcp{en tiempo real}, en donde platicaremos sobre los videos/lecturas, para que ustedes hagan preguntas.\pause
 \item Para entrar a esta clase, \tcb{será requisito haber visto los videos e intentado 1 vez los quizzes correspondientes}. \tcr{¡¡Si no los viste, no puedes entrar!!}
\end{enumerate}
\end{frame}

\section{Calificaciones}

\begin{frame}\frametitle{¿Cómo voy a calificar?}
	\begin{itemize}
		\item Quizzes: 20\%
		\item Tareas: 35\%
		\item Experimento arquitectura: 20\%
		\item Proyecto individual y exposición: 25\%
		\item \scriptsize{Lecturas y Tutoriales: 10\% extra}
	\end{itemize}
	\pause
	La calificación será \tcg{la media geométrica pesada}\footnote{$\sqrt[100]{Q^{20}T^{35}E^{20}P^{25}} + 0.1 L$} de las 4 primeras + 10\% de la última. Es decir, si sacas 0 en UNA de las primeras, sacas 0 en total (o posiblemente hasta 1 de calificación).
\end{frame}

\begin{frame}\frametitle{Quizzes (20\%)}
    \begin{enumerate}
     \item Serán sencillos, de opción múltiple, si viste los videos.\pause
     \item Puedes intententarlos tantas veces como quieras. \pause
     \item Cada vez que los intentas te dirá qué tuviste mal y así, para que lo hagas mejor a la siguiente.\pause
     \item \tcr{Importante:} Para tener derecho a calificación, \tcg{debes sacar 10 en todos los quizzes sin excepción}.\pause
     \item \tcr{Importante:} Para poder entrar a la sesión en vivo, \tcg{debes haber intentado por lo menos una vez el quiz}. No importa cuánto hayas obtenido.
    \end{enumerate}
\end{frame}


\begin{frame}\frametitle{Tareas (35\%)}
\begin{itemize}
 \item Las tareas consistirán básicamente en entregar libretas de jupyter en donde harán cosas bien padres!\pause
 \item En la tarea \tcg{se vale consultar con profesores, compañeros, etc.}, pero absolutamente todo lo que entregues debe ser escrito por ti (nunca copy-paste!). \tcr{NO} se vale mostrarle tu código a nadie. Se vale enviar documentación, o decirle ``ve la documentación de torch.sigmoid'' o lo que sea.\pause
 \item Usualmente serán modificaciones pequeñas a lo que les mostraré en los videos.\pause
 \item No les pasaré las libretas, solo screenshots de ellas. ¡Así tendrán que copiar y no dar CTRL-C, CTRL-V!
\end{itemize}
\end{frame}

\begin{frame}\frametitle{Experimento Arquitectura (20\%)}
Es como una tarea grande y ``abierta'', en el sentido de que la idea es que aprendan a experimentar.\pause
 \begin{itemize}
  \item Les proporcionaré datos para clasificación de imágenes, y pondré algunas restricciones. \pause
  \item Ustedes harán todos los experimentos que se les ocurran, tratando de alcanzar la máxima \textit{accuracy} (precisión) que puedan.\pause
  \item Voy a calificar cuánto experimentaste y qué ideas tuviste, NO solo a cuánta precisión llegaste. Así que debes reportar todo, lo que funcionó y lo que no.\pause
  \item Sin embargo, habrá premios \textit{virtuales}\footnote{En años pasados les di chocolates, pero ahora... me los comeré yo en su honor.} a los que lleguen a más precisión.
 \end{itemize}
\end{frame}

\begin{frame}\frametitle{Exposición/Proyecto (25\%)}
 \begin{itemize}
  \item Al final del curso cada quien deberá hacer algún \tcg{proyecto individual} original, de algo que les interese y exponerlo (e.g. mandar video de youtube).\pause
  \item No tiene que ser algo innovador o en ``estado del arte'', pero sí algo interesante y que NO hayamos hecho en clase algo igual.\pause
  \item Más detalles ya después.
 \end{itemize}
\end{frame}


\begin{frame}\frametitle{Tutoriales y Lecturas (10\%)}
Es muy importante mantenerse al día en esta área. Por eso deben constantemente leer artículos, ver videos de youtube, hacer tutoriales, etcétera.\pause
 \begin{itemize}
  \item Cada semana reportarás exactamente qué artículos, videos o tutoriales viste, y qué aprendiste de cada uno.\pause
  \item Cada semana les pondré algunas lecturas opcionales, pero la idea es que tú busques otras. \pause
  \item En base a cuánto hayas leído, etc. te daremos hasta 1 punto extra en la calificación final.\pause
  \item Les recomiendo consumir al menos 2 o 3 por semana. Yo más o menos leo o veo uno al día, aunque no todos los días. Más o menos como 5 por semana.
 \end{itemize}
\end{frame}

\begin{frame}\frametitle{¿De donde sacar contenido?}
 \begin{itemize}
  \item Reddit (r/MachineLearning, r/deeplearning)
  \item Youtube (two minute papers, 3blue1brown, dotcsv, etc.)
  \item Medium (toward data science, etc.)
  \item kdnuggets
  \item distill
  \item thegradient
 \end{itemize}
\end{frame}

% \begin{frame}\frametitle{Programa}
%   Parte I: Probabilidad Discreta
%   \begin{enumerate}
%     \item Espacios de probabilidad
%     \item Probabilidad combinatoria
%     \item Variables aleatorias
%     \item Operadores: Esperanza, Varianza, Covarianza, etc.
%     \item Redes Bayesianas
%   \end{enumerate}
%   
%   Parte II: Probabilidad Continua
%   \begin{enumerate}
%     \item Diferencias entre discreta y contínua
%     \item Espacios de probabilidad contínuos
%     \item Variables aleatorias, distribuciones
%     \item Operadores: Esperanza, Varianza, Covarianza, etc.
%     \item Teorema del límite central
%     \item Desigualdades
%   \end{enumerate}
% 
%   Si da tiempo, también veremos ``Parte III: Pensamiento Crítico'', aunque no es parte del programa (semi)oficial.
% \end{frame}


\end{document}
